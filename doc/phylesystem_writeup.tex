\documentclass[a4paper,10pt]{article}
\usepackage[utf8]{inputenc}

%opening
\title{Using a git-based datastore for community curated phylogenies}
\author{Emily Jane McTavish, Mark T. Holder ... ??}

\begin{document}

\maketitle

\begin{abstract}
Motivation:
The lack of an accessible, reusable, and well curated database has hindered biological research into evolutionary questions.
The Open Tree of Life project is collating and synthesizing hundreds of phylogenies into a unified tree of life. 
The base data for for this project are community contributed phylogenies. 
Via the OpenTree curator tool, these phylogenies are updated to reflect consistent tip names across studies, and accurate rooting. 
Results:
Using a git-based datastore to hold these phylogenies automatically version controls data as updates are made. 
In addition, hosting this datastore on GitHub provides a straightforward and familiar method for researchers to access these data. 
We have developed an API ``Phylesystem'' to implement wrappers around git, interaction with github, and to return readable diffs for complex data structures.
Availabilty:
Source code for the API is available ...
The git database is available ...
\end{abstract}

\section{}
The Open Tree of Life project is creating a synthetic phylogeney, incorporating published phylogenies from across the tree of life.
We are using a git based datastore, mirror to github, to simulataneously track study curation and disseminate the phylogenies.

Outline:
 Introduction:
 While characterizing and systematizing relationships among species has been a goal of biologists since Linneaus \cite{Linneaus}, 
 the recent revolution in availability of molcular data has greatly increased the rate at which phylogenetic ananlyses are bring performed.
 The National Science foundation has funded many 'tree of life' efforts to resolve phylogenetic reationships among taxa within specific groups.
 The Open tree of life project 
 
 
  Background:
     TOL projects
     Tree base and Phylografter
  
  Purpose:
      - Accessible data store
      - Version control
          - OTU mapping
          - rerooting
          - many curators!, possibly editing studies at the same time!
          - long term usef, ideally.

          
 Implementation of database (structure):
      - FIGURE TO REPRESENT!
      - File Formats
         - Explain Nexson... UGH
         
      - Sharded git repository
          Describe full structure 
          
      - External library to index studies (is OTI described somewhere?)
      - peyotl...
      
      Curation via app:
        - main repo on server -> mirror -> github
            ??? do pulls from github happen? wholes cycle *can* be reversed.
        - dealing with WIP branches.
        *Even conflicts that Git may find mergable"*
        return merge conflicts to user

 Discussion:
   Generalizability:
      Git may be a good option for any non-static data store.
      
  Future directions

  
  
  --------------------------------------------------------------------
Key Points:
Pros
 - Versioned
 - Familiar to users
 - Comes with a lot of great machineray already enabled

diffing doesn't always behave well.
simple data structures behave oddly...

Meaningful merges
 -- immutable
 -- can't

Is this model applicable for database store

General model based decisions:
- Decisions that studies won't move.

- given the study ID what shard will it be in.

- scales better in general.

- Data is so open.



 The current (as of May) data set
 Community contributed phylogenies

 - 6745 trees from 2914 published studies
 - 1188 trees from 991 studies partly curated 
 - 335 trees from 327 studies completely curated and included in the synthetic tree.

 The problem:
 - Large data set: Thousands of phylogenies, and always growing (hopefully!)
 - Each phylogeny requires some hand curation, often by multiple people
 - Need to be readily accessible, and editable by interested researchers

Curation

 Potential data store options:
 - SQL database
 - Mongo, couchDB
 - git/github


 We chose git!

 - Trees and annotations by study in Nexson format  
(JSON serialization of NeXML)
 - Whole datastore is a git repo!  
 

 Curation
- Work in progress branch is created upon curation  
- If study hasn't been edited by someone else, changes are automatically merged.  
- Otherwise, merged changes are returned to curator to accept or reject
- Updates pushed to GitHub after each commit
 Features
 - Tracking curation attribution  
- Some subjective choices, edits made by many in the community over time

 Curation

 Features
 These trees are the backend for OpenTree showpiece  
 the synthetic tree!
 - but also a useful datastore for other researchers
 - Repo is hosted on GitHub, entire data store can be easily cloned and updated
 - Anyone can easily download all the data!
 
 Features
- Hosting on Github  
- Free  
- Familar to  many in the field


 Potential issues:
 - Phylogenies are hard to diff - e.g. rerooting changes everything!
 - Nexson are not a line based format
 - Repo size limits on github


 In the future:

 - Semantic diffs  
 - Pull requests
 
  
Generality:
Is a git-based datastore right for your project?

 - Maybe! Any of the OpenTree software team are happy to chat about pros and cons.

NSF AVATOL \#1208809  



\end{document}
