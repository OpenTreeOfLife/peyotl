\documentclass[a4paper,10pt]{article}
\usepackage[utf8]{inputenc}

%opening
\title{Using a git-based datastore for community curated phylogenies}
\author{}

\begin{document}

\maketitle

\begin{abstract}
The Open Tree of Life project is collating and synthesizing hundreds of phylogenies into a unified tree of life. 
The base data for for this project are community contributed phylogenies. 
Via the OpenTree curator tool, these phylogenies are updated to reflect consistent tip names across studies, and accurate rooting. 
Using a git-based datastore to hold these phylogenies automatically version controls data as updates are made. 
In addition, hosting this datastore on GitHub provides a straightforward and familiar method for researchers to access these data. 
The talk will cover our implementation of wrappers around git, interaction with github, and developing tools to return readable diffs for complex data structures.
\end{abstract}

\section{}
The Open Tree of Life project is creating a synthetic phylogeney, incorporating published phylogenies from across the tree of life.
We are using a git based datastore, mirror to github, to simulataneously track study curation and disseminate the phylogenies.


Key Points:
Pros
 - Versioned
 - Familiar to users
 - Comes with a lot of great machineray already enabled

diffing doesn't always behave well.
simple data structures behave oddly...

Meaningful merges
 -- immutable
 -- can't

Is this model applicable for database store

General model based decisions:
- Decisions that studies won't move.

- given the study ID what shard will it be in.

- scales better in general.

- Data is so open.



\end{document}
